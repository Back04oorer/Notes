\documentclass[12pt]{article}

%%% UPDATE THESE PARAMETERS
\newcommand{\sid}{530157791}
\newcommand{\assignment}{1}
\newcommand{\uos}{comp2022}
%%%

\NeedsTeXFormat{LaTeX2e}
\usepackage[osf]{mathpazo}
\usepackage[svgnames]{xcolor}
\usepackage[T1]{fontenc}
\usepackage{amsmath,amsthm,amsfonts,amssymb,mathtools}
\usepackage{hyperref,url}
\usepackage[margin=2.7cm,a4paper]{geometry}
\usepackage{tasks}
\usepackage{xstring}
\usepackage[tikz]{mdframed}
\usepackage{environ}
\usepackage{etoolbox}
\usepackage{fourier-orns}
\usepackage{kvoptions}
\usepackage[]{units}
\usepackage{url}%
\usepackage[normal]{subfigure}

% math config

\DeclarePairedDelimiter\ceil{\lceil}{\rceil}
\DeclarePairedDelimiter\abs{\lvert}{\rvert}
\DeclarePairedDelimiter\set{\{}{\}}


% headers

\usepackage{fancyhdr}
\addtolength{\headheight}{2.5pt}
\pagestyle{fancy}
\fancyhead{} 
\fancyhead[L]{\sc \uos} 
\renewcommand{\headrulewidth}{0.75pt}

% update these headers
\fancyhead[C]{\sc SID: \sid}
\fancyhead[R]{Assignment \assignment}


% pseudo-code typesetting configuration
\usepackage[tikz]{mdframed}
\usepackage[noend]{algpseudocode}

\newenvironment{pseudocode}
{
  \mdframed
  \algorithmic[1]
}
{
  \endalgorithmic
  \endmdframed
}

% solution 

\newcommand{\solution}[1]{\noindent \textbf{Solution #1.}}


\begin{document}

\solution{4.1} 
The idea is that each state in \( Q' \) not only stores the state in \( Q \) but also tracks the current letter position (1st, 2nd, 3rd). Specifically, we extend each state \( q \in Q \) to \( (q,0) \), \( (q,1) \), and \( (q,2) \), representing every first, second, and third letter. For the first and second characters, we only update the number in the tuple. For example, \( (q_0,0) \) becomes \( (q_0,1) \) after receiving the first character. When the third letter is received (i.e., the number in the tuple is 2), the number resets to 0, and \( q_0 \) transitions according to the original transition function in $\delta$. For example:
\[
(q_0,2) \xrightarrow{a} (q_1,0)
\]
The accept states are set to \( (q_f, 0) \), \( (q_f, 1) \), and \( (q_f, 2) \) for all \( q_f \in F \).\\



\solution{4.2} 
For DFA B, we have DFA B $= (Q',\Sigma,q_0',\delta',F')$:

\begin{itemize}
  \item $Q' = Q \times \{0,1,2\}$
  \item $q_0' = (q_0,0)$
  \item $\delta' = \delta'((q,n),l) = \left\{
      \begin{array}{ll}
      \text{$(q,n+1)$} & \text{$n = 0,1$} \\
      \text{$(\delta(q,l),0)$} & \text{$n = 2$}
      \end{array}
      \right.$, $q \in Q$, $l \in \Sigma$
  \item $F' = (q,n),q \in F,n \in \{0,1,2\}$

\end{itemize}




\end{document}
