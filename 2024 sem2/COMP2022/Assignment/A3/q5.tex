\documentclass[12pt]{article}

%%% UPDATE THESE PARAMETERS
\newcommand{\sid}{530157791}
\newcommand{\assignment}{3}
\newcommand{\uos}{comp2022}
%%%

\NeedsTeXFormat{LaTeX2e}
\usepackage[osf]{mathpazo}
\usepackage[svgnames]{xcolor}
\usepackage[T1]{fontenc}
\usepackage{amsmath,amsthm,amsfonts,amssymb,mathtools}
\usepackage{hyperref,url}
\usepackage[margin=2.7cm,a4paper]{geometry}
\usepackage{tasks}
\usepackage{xstring}
\usepackage[tikz]{mdframed}
\usepackage{environ}
\usepackage{etoolbox}
\usepackage{fourier-orns}
\usepackage{kvoptions}
\usepackage[]{units}
\usepackage{url}%
\usepackage[normal]{subfigure}

% math config

\DeclarePairedDelimiter\ceil{\lceil}{\rceil}
\DeclarePairedDelimiter\abs{\lvert}{\rvert}
\DeclarePairedDelimiter\set{\{}{\}}


% headers

\usepackage{fancyhdr}
\addtolength{\headheight}{2.5pt}
\pagestyle{fancy}
\fancyhead{} 
\fancyhead[L]{\sc \uos} 
\renewcommand{\headrulewidth}{0.75pt}

% update these headers
\fancyhead[C]{\sc SID: \sid}
\fancyhead[R]{Assignment \assignment}


% pseudo-code typesetting configuration
\usepackage[tikz]{mdframed}
\usepackage[noend]{algpseudocode}

\newenvironment{pseudocode}
{
  \mdframed
  \algorithmic[1]
}
{
  \endalgorithmic
  \endmdframed
}

% solution 

\newcommand{\solution}[1]{\noindent \textbf{Solution #1.}}


\begin{document}
\solution{5.1} Assume that the language is decidable. Then there exists a Turing machine \(D\) that can decide whether the given input  $M, x$ is in the language or not. For the input $M, x$, we can construct a new Turing machine \( M' \) with the following behavior:

\begin{itemize}
  \item \( M' \) simulates \( M \) on the input string \( x \);
  \item If \( M \) halts during the simulation, \( M' \) will continue running until exactly \( 77n \) steps have been taken;
  \item If \( M \) does not halt, \( M' \) will continue running.
\end{itemize}
By constructing \( M' \), we can solve the Halting Problem. If \( D \) accepts \(  M', x  \), it means that \( M \) halts on \( x \), and \( M' \) halts in exactly \( 77n \) steps. If \( D \) rejects, it means that \( M \) diverges on input \( x \).Therefore, if such a Turing machine \( D \) existed that could decide whether \( M' \) halts in exactly \( 77n \) steps, we would be able to use it to solve the Halting Problem, which is known to be undecidable. This leads to a contradiction because the existence of \( D \) would imply that the Halting Problem is decidable. Hence, the language $\{ M, x  : M \text{ halts on } x \text{ in exactly } 77n \text{ steps for some integer } n > 0 \}$ must be undecidable.


\end{document}
