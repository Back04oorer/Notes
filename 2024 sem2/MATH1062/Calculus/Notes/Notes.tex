\documentclass{article}
\usepackage{soul}
\usepackage{xcolor}
\sethlcolor{pink}
\usepackage{amsmath}
\usepackage{amssymb}
\usepackage{cancel}
\usepackage{graphicx}
\usepackage{CJKutf8} 
\usepackage{tikz}


\title{MATH1023/MATH1062 Calculas}
\author{Usyd Mingyuan Ba}
\date{\today}

\begin{document}

\maketitle

% \section{Week1}
%   \subsection{Differential Equation}
%   \begin{enumerate}

%     \item \textbf{Differential Equation(DE)}: A differential equation (DE) is a mathmatical equation that relates some \hl{function with its derivatives}

%     \item \textbf{Order}: The order of a differential equation equals to a \hl{highest derivative} occuring in it.
%       \begin{itemize}
%         \item $\frac{dy}{dx} = -ky$ has order \hl{$1$}
%         \item $\frac{dy}{dx} = y^{18} + \frac{d^5y}{dx^2}y + x^2$ has order \hl{5}
%       \end{itemize}

%     \item \textbf{Standard Form}: The standard form of a \hl{first-order differential equation} is
%       \begin{center}
%         $\frac{dy}{dx} = f(x,y)$
%       \end{center}

%     \item \textbf{General Solution}: A \hl{general solution} is a solution incoprating all constants of integration.

%     \item \textbf{Initial Condition}: An initial condition is a pair $(x_0,y_0)$ such that $y(x_0) = y_0$


%   \end{enumerate}


% \section{Week2}
%   \subsection{Direction Field}
%   \begin{enumerate}
%     \item \textbf{Definition}: A direction field of a DE
%       \begin{center}
%         $y' = f(x,y)$
%       \end{center}
%     consists of a grid of short line segments with slope $f(a,b)$ drawn at points $(a,b)$.So the line segment at $(a,b)$ is \hl{tangent} to any solution passing through $(a,b)$

%     \item \textbf{Example}:Draw some solution curves on the given direction field for the DE:
%       \begin{center}
%         $y' = xy$
%       \end{center}
%     \includegraphics[width=\linewidth]{Graphs/direction_field.png}
%   \end{enumerate}

%   \subsection{Separable equations}

%   \begin{enumerate}
%     \item \textbf{Definition}: A first-order DE $y' = f(x,y)$ is called \hl{separable} if there are functions $g(x)$ and $h(y)$ such that \hl{$f(x,y) = g(x)h(y)$}, so a separable DE can be written
%     \begin{center}
%       \hl{$y' = g(x)h(y)$}
%     \end{center}

%     \item \textbf{Goal}: We want to find a method for solving separable DEs

%     \item \textbf{Method}: We can solve a separable DE:
%       \begin{center}
%         $\frac{dy}{dx} = g(x)h(y)$
%       \end{center}
%     by separating variables.\\

%     Dividing both sides by $h(y)$ gives 
%       \begin{center}
%         $\frac{1}{h(y)} \frac{dy}{dx} = g(x)$
%       \end{center}

%     Intergrating both sides gives:
%       \begin{center}
%         $\int \frac{1}{h(y)} = \int g(x) dx$ 
%       \end{center}

%     If we can find antiderivatives $H(y)$ for $\frac{1}{h(y)}$ and $G(x)$ for $g(x)$, then we have
%       \begin{center}
%         $H(y) = G(x) + C$ 
%       \end{center}
%   \end{enumerate}

%   \section{Week3}
%     \subsection{Modelling Population Growth}
%       \begin{enumerate}
%         \item \textbf{Constant Growth}: This occurs when the population $x$ increases at a constant rate. The DE is 
%           \begin{center}
%             $\frac{dx}{dt} = k$
%           \end{center}
%           where k is constant

%         \item \textbf{Exponential Growth}: The exponential growth model assumes the growth rate is proportional to the size of the population.

%         The general form of a DE modelling exponential growth is 
%           \begin{center}
%             $\frac{dx}{dt} = kx$
%           \end{center}
%           where k is constant

%         \item \textbf{Logistic Growth}: Exponential growth is \hl{not} a realistic growth model for all values of $t$. \hl{A small animal population} with unlimited resources of food and space \hl{may show exponential growth initially}

%         As the population gets larger there will be food shortages, overcrowding, and other factors that \hl{slow down the growth rate}.

%         \hl{The growth rate k should decrease as the population x increases.}

%         Since k is no longer constant, we write $k=g(x)$, so the DE becomes 
%           \begin{center}
%             $\frac{dx}{dt} = g(x)x$
%           \end{center}

%         A small population can growth exponentially, se we want \hl{$g(x) \approx k$} when \hl{$x \approx 0$}. But \hl{as x increases $g(x)$ should decrease.}

%         The simplest formula with this behaviour is 
%           \begin{center}
%             $g(x) = k - ax$
%           \end{center}

%         So the DE becomes 
%           \begin{center}
%             $\frac{dx}{dt} = (k-ax)x$
%           \end{center}

%         We introduce a new constant $b = \frac{k}{a}$ so
%           \begin{center}
%             $(k-ax)x = ax (\frac{k}{a}-x) = ax(b-x)$
%           \end{center}

%         Let $\frac{b}{a} = b$, the logistic DE is then given by

%           \begin{center}
%             $\frac{dx}{dt} = ax(b-x)$
%           \end{center}
%       \end{enumerate}

%   \section{week4}

%     \subsection{First-order linear DEs}
%       \begin{enumerate}
%         \item \textbf{First-order linear differential equation}: A first-order linear differential equation is a DE of the form:
%           \begin{center}
%             $\frac{dy}{dx} + p(x) y = q(x)$
%           \end{center}
%           \textbf{$\frac{dy}{dx}$ and $y$ occur only linearly}

%         \item \textbf{How to solve first-order linear DEs ?}: The idea is multiplying the DE by a function $r(x)$ give:
%           \begin{center}
%             $r(x) \frac{dy}{dx} + r(x)p(x) = r(x)q(x)$
%           \end{center}

%           If we can find $r(x)$ such that:
%           \begin{center}
%             $r(x) \frac{dy}{dx} + r(x)p(x) =$ \hl{$ \frac{d}{dx}(r(x)y(x))$}
%           \end{center}

%           then the DE becomes:
%           \begin{center}
%             \hl{$\frac{d}{dx}(r(x)y(x))$} $= r(x)q(x)$
%           \end{center}

%           Integrating with respect to x gives:
%           \begin{center}
%             $\int \frac{d}{dx}(r(x)y(x))dx = \int (r(x)q(x))dx$ \\
%             $\rightarrow$ \\
%             $r(x)y(x) = \int r(x)q(x)dx + C$

%           \end{center}

%           so the general solution is

%           \begin{center}
%             $y = \frac{1}{r(x)} [\int r(x)q(x) dx + C]$
%           \end{center}

%         \item \textbf{Integrating factor}: The function
%           \begin{center}
%             $r(x) = e^{\int p(x) dx}$
%           \end{center}
%           is an intergrating factor for the first-order linear DE

%           \begin{center}
%             $\frac{dy}{dx} + p(x)y = q(x)$
%           \end{center}

%         \item \textbf{General Solution } the general solution of the DE is 
%           \begin{center}
%             $y = \frac{1}{r(x)} [\int r(x)q(x) dx + C]$
%           \end{center}

%       \end{enumerate}


%   \section{week5}

%     \subsection{Higher order differential equations}

%       Higher order DEs involve higher order derivatives.For example, the DE:

%       \begin{center}
%         $\frac{d^2 y}{dx^2} + f(x,y) \frac{dy}{dx} = g(x,y)$
%       \end{center}
%       is a \hl{second-order differential} equation.

%       \begin{enumerate}
%         \item Solving higher-order DEs is harder.
%         \item The general solution of a second-order DE has 2 degrees of freedom, so needs two initial conditions.
%         \item The general solution of an \hl{nth-order DE has n degrees of freedom}, so \hl{needs n initial conditions}.
%       \end{enumerate}

%     \subsection{Second-order linear DEs with constant coefficients}
%       \begin{enumerate}
%         \item \textbf{Definition} A \hl{second-order linear differential equation} is a DE that can be expressed in the form:
%         \begin{center}
%           $\frac{d^2y}{dx} + f_1(x)\frac{dy}{dx} + f_0(x)y = g(x)$
%         \end{center}

%         \hl{The DE is linear in $y$ and its derivatives.}

%         \item \textbf{homogeneous/inhomogeneous}
%             \begin{itemize}
%               \item The DE is homogeneous if $g(x) = 0$
%               \item The DE is inhomogeneous if $g(x) \neq 0$
%             \end{itemize}

%         If \hl{$g(x) = 0,f_0(x) = a, f_1(x) = b$ for $a,b \in \mathbb{R}$}, then we have a \hl{homogeneous second-order linear differential equation} with constant coefficient:

%         \begin{center}
%           $\frac{d^2 y}{dx^2} + a \frac{dy}{dx} + by = 0$
%         \end{center}

%         \item Solve the above DE : 
%         \begin{center}
%           $\frac{d^2 y}{dx^2} + a \frac{dy}{dx} + by = 0$
%         \end{center}

        
%         \begin{itemize}
%           \item \textbf{Observation 1}: y is a linear combination of its first two derivatives, so we try:
%           \begin{center}
%             $y(x) = e^{mx}$
%           \end{center} 

%           We have

%           \begin{center}
%             $\frac{dy}{dx} = me^{mx}, \frac{d^2y}{dx^2} = m^2e^{mx}$
%           \end{center} 

%           \item \textbf{Observation 2}: Find m such that $y = Ce^{mx}$ satisfies the DE
%           \begin{center}
%             $\frac{d^2y}{dx^2} + a\frac{dy}{dx} + by = 0$
%           \end{center}
%           substituting y and its derivatives we get:
%           \begin{center}
%             $Cm^2e^{mx} + aCme^{mx} + bCe^{mx} = 0$\\
%             $\Rightarrow Ce^{mx}(m^2+am+ b) = 0$\\
%             $\Rightarrow m = \frac{-a \pm \sqrt{a^2 - ab}}{2} $
%           \end{center}

%           So we have  2 solutions

%           \begin{center}
%             $m_1 = \frac{-a + \sqrt{a^2 - ab}}{2}, m_2 = \frac{-a - \sqrt{a^2 - ab}}{2}$
%           \end{center}       

%           \item \textbf{Observation 3:} Show that if $m = m_1,m_2$ are solutions of $m^2 + am +b$,then \hl{$y = C_1e^{m_1x} + C_2e^{m_2x}$}, satisfies the DE

%           \begin{center}
%             $\frac{d^2y}{dx^2} + a\frac{dy}{dx} + by = 0$
%           \end{center}

%           we have 
%           \begin{center}
%             $y = C_1e^{m_1x} + C_2e^{m_2x}$\\
%             $\Rightarrow \frac{dy}{dx} = m_1C_1e^{m_1x} + m_2C_2e^{m_2x}$\\
%             $\Rightarrow \frac{d^2y}{dx^2} = m_1^2C_1e^{m_1x} + m_2^2C_2e^{m_2x}$
%           \end{center}

%           substituting into the DE we get

%           \begin{center}
%             $2$\\
%             $\Rightarrow \frac{dy}{dx} = m_1C_1e^{m_1x} + m_2C_2e^{m_2x}$\\
%             $\Rightarrow \frac{d^2y}{dx^2} = m_1^2C_1e^{m_1x} + m_2^2C_2e^{m_2x}$
%           \end{center}

%           substituting into the DE we get 

%           \begin{align*}
%             \frac{d^2 y}{dx^2} + a \frac{dy}{dx} + by &= m_1^2 C_1 e^{m_1 x} + m_2^2 C_2 e^{m_2 x} + a \left( m_1 C_1 e^{m_1 x} + m_2 C_2 e^{m_2 x} \right) + b \left( C_1 e^{m_1 x} + C_2 e^{m_2 x} \right) \\
%             &= C_1 e^{m_1 x} \left( m_1^2 + a m_1 + b \right) + C_2 e^{m_2 x} \left( m_2^2 + a m_2 + b \right) \\
%             &= 0
%           \end{align*}


%           \item \textbf{formal solution:} 
%             We now have a good candidate for a general solution of the DE:
%             \begin{center}
%               $y =C_1e^{m_1x} + C_2e^{m_2x}$ 
%             \end{center}
%             Where $m_1 = \frac{-a + \sqrt{a^2-4b}}{2}$,$m_2 = \frac{-a - \sqrt{a^2-4b}}{2}$ are solutions of $m^2 + am +b = 0$. We have 3 cases to consider:



%             \begin{itemize}
%                 \item \textbf{Case 1}: For \hl{$a^2 > 4b$} we have 2 distinct real solutions
%                 \begin{center}
%                   $m_1 \neq m_2, m_1,m_2 \in \mathbb{R}$
%                 \end{center}
%                 The general solution is 
%                 \begin{center}
%                   $y =C_1e^{m_1x} + C_2e^{m_2x}$
%                 \end{center}


%                 \includegraphics[width=\linewidth]{Graphs/w5_1.png}


%                 \item \textbf{Case 2}: For \hl{$a^2 < 4b$} we have 2 distinct complex solutions:
%                 \begin{center}
%                   $m_1,m_2 = \frac{-a \pm \sqrt{a^2-4b}}{2} = \frac{-a \pm 2ik}{2} = - \frac{a}{2} \pm ik$\\
%                   where \hl{$k = \frac{1}{2} \sqrt{4b-a^2} > 0$}
%                 \end{center}
%                 Using Euler's formula:
%                 \begin{center}
%                   $e^{ikx} = cos(kx) + isin(kx)$
%                 \end{center}
                
%                 We have

%                 \[
%                 y = C_1 e^{m_1 x} + C_2 e^{m_2 x}
%                 \]

%                 \[
%                 = C_1 e^{\left(-\frac{a}{2} + ik\right)x} + C_2 e^{\left(-\frac{a}{2} - ik\right)x}
%                 \]

%                 \[
%                 = e^{-\frac{a}{2}x} \left( C_1 e^{ikx} + C_2 e^{-ikx} \right)
%                 \]

%                 \[
%                 = e^{-\frac{a}{2}x} \left( C_1 \left(\cos(kx) + i\sin(kx)\right) + C_2 \left(\cos(kx) - i\sin(kx)\right) \right)
%                 \]

%                 \[
%                 = e^{-\frac{a}{2}x} \left( \left(C_1 + C_2\right) \cos(kx) + i \left(C_1 - C_2\right) \sin(kx) \right)
%                 \]

%                 \[
%                 = e^{-\frac{a}{2}x} \left( D_1 \cos(kx) + D_2 \sin(kx) \right)
%                 \]

%                 So the general solution is:
%                 \hl{$y = e^{-\frac{a}{2}x} \left( D_1 \cos(kx) + D_2 \sin(kx) \right)$}

%                 \includegraphics[width=\linewidth]{Graphs/direction_field.png}


%                 \item \textbf{Case 3}: For $a^2 = 4b$ we have 1 real solution:
%                   \begin{center}
%                     \( m_1 = m_2 = \frac{-a \pm \sqrt{a^2 - 4b}}{2} = -\frac{a}{2} \)
%                   \end{center}
%                   Our solution becomes

%                   \[
%                   y = C_1 e^{-\frac{a}{2}x} + C_2 e^{-\frac{a}{2}x}
%                   \]

%                   \[
%                   = \left( C_1 + C_2 \right) e^{-\frac{a}{2}x}
%                   \]

%                   \[
%                   = D e^{-\frac{a}{2}x}
%                   \]

%                   Here, \( D \) is a constant (\( D = C_1 + C_2 \)), which means we only have \hl{1 degree of freedom, so this is not a general solution}.

%                   \newpage

%                   We look for a general solution of the form
%                   \begin{center}
%                     $y = f(x)e^{-\frac{a}{2}x}$
%                   \end{center}

%                   Substituting \( y \) and its derivatives into the differential equation (DE)

%                   \[
%                   \frac{d^2y}{dx^2} + a\frac{dy}{dx} + by = 0
%                   \]

%                   gives

%                   \[
%                   e^{-\frac{a}{2}x} \left( f''(x) + \frac{1}{4}(4b - a^2)f(x) \right) = 0 \quad \text{(exercise)}
%                   \]

%                   Since \( e^{-\frac{a}{2}x} \neq 0 \),

%                   \[
%                   f''(x) = \frac{1}{4}(a^2 - 4b)f(x) = 0
%                   \]

%                   which implies

%                   \[
%                   f'(x) = C_2
%                   \]

%                   \[
%                   f(x) = C_2 x + C_1
%                   \]

%                   Hence, the general solution is

%                   \[
%                   y = (C_1 + C_2 x) e^{-\frac{a}{2}x}
%                   \]


%             \end{itemize}
%         \end{itemize}

%       \end{enumerate}

%   \newpage

%   \section{week6}
%     % \begin{CJK}{UTF8}{gbsn} % 开始中文环境,gbsn是简体中文宋体
%     % 你好,世界! % 在这里可以输入中文
%     % \end{CJK}
%     \subsection*{Simple harmonic motion}
%     \begin{itemize}
%       \item Periodic bhaviour \hl{without} damping is modelled by the DE
%         \begin{center}
%         $\frac{d^2x}{dt^2} + bx = 0,b>0$\\
%         or\\
%         $\ddot{x} + w_{0}^2x = 0$
%         \end{center}

%       \item We can express the solution as 
%         \begin{center}
%           $x = Acos(w_0t + \phi)$
%         \end{center}
%         \begin{itemize}
%           \item $A$ = amplitude
%           \item $w_0$ = frequency
%           \item $\phi$ =  phase
%           \item $T = \frac{2 \pi}{w_0}$ = period
%         \end{itemize}
%     \end{itemize}

%     \subsection*{Damped harmonic oscillator}
%     \begin{itemize}
%       \item Periodic behaviour \hl{with} damping is  modelled by the DE:
%         \begin{center}
%           $\frac{d^2x}{dt^2} + a \frac{dx}{dt} + bx = 0$
%         \end{center}

%         with $a = 2 \gamma$,$b = \omega_0^2$, or
%         \begin{center}
%           $\ddot{x} + 2 \gamma \dot{x} + \omega_0^{2}x = 0$
%         \end{center}

%       \item The characteristic equation is 
%         \begin{center}
%           $m^2 + am + b = 0$
%         \end{center}
%         which has solution 
%         \begin{center}
%           $ m = \frac{-a \pm \sqrt{a^2 - 4b}}{2} = - \gamma \pm \sqrt{\gamma^2 - \omega_0^2}$
%         \end{center}

%     \end{itemize}

%     \subsection*{Inhomogeneous second-order linear DEs with constant coefficients}

%     \begin{itemize}
%       \item An \hl{inhomogeneous scond-order linear differential equation} with constant coefficients is a DE that can be expressed in the form:
%       \begin{center}
%         $\frac{d^2y}{dx^2} + a\frac{dy}{dx} + by = g(x)$
%       \end{center}

%       \item \textbf{theorem}: Let $y_p(x)$ be a particular solution of an \hl{inhomogeneous linear DE} and let $y_h(x)$ be the \hl{general solution} of the corresponding homogeneous DE. Then the general solution of the inhomogeneous DE is the
%       \begin{center}
%         $y(x) = y_h(x) + y_p(x)$
%       \end{center}

%       \item \hl{systems of first-order linear DEs with constant coefficients}: A system of two first-order DEs with constant coefficients has the form:

%       \begin{center}
%         $\frac{dx}{dt} = ax + by$ \texttt{(*)},\\
%         $\frac{dy}{dt} = cx + fy$ \texttt{(**)}
%       \end{center}

%       to solve this system, we follow the following steps:
%       \begin{enumerate}
%         \item Differentiate \texttt{(*)}
%           \begin{center}
%             $\frac{d^2x}{dt^2} = a\frac{dx}{dt} +$ \hl{$b \frac{dy}{dt}$} \texttt{(I)}
%           \end{center}

%         \item Substitude the right hand side of of \texttt{(**)} into \texttt{(I)}
%           \begin{center}
%             $\frac{d^2x}{dt^2} = a\frac{dx}{dt} +$ \hl{$b(cx + fy)$} \texttt{(II)}
%           \end{center}

%         \item Rearrange \texttt{(*)} to make y the subject
%           \begin{center}
%             $y = \frac{1}{b} (\frac{dx}{dt} - ax)$ \texttt{(III)}
%           \end{center} 
%         \item Substitude the right hand side of \texttt{(III)} into \texttt{(II)} 
%           \begin{center}
%             $\frac{d^2x}{dt^2} = a \frac{dx}{dt} + b(cx + 
%             \frac{f}{b}(\frac{dx}{dt}-ax)) \rightarrow \frac{d^2x}{dt^2} = (a+f)\frac{dx}{dt} + (bx - af)x$
%           \end{center}

%         \item Solve the DE   
%           \begin{center}
%             $\frac{d^2x}{dt^2} - (a+f)\frac{dx}{dt} - (bc-af)x = 0$ for x.
%           \end{center}
%         \item Substitute x into \texttt{(**)} and solve the \hl{first-order linear DE} for y
%         \begin{center}
%           $\frac{dy}{dt} = cx + fy \rightarrow \frac{dy}{dt} + p(t)y = q(t)$

%         \end{center}       
%       \end{enumerate}
%     \end{itemize}

% \section{Week7}
%   \subsection*{2-dimensional plane}
%     \begin{itemize}
%       \item The \textbf{2-dimensional plane}, often called the $(x,y)$-plane, can be represented by the set
%       \[
%         \mathbb{R}^2 = \{(x,y) \mid x,y \in \mathbb{R}\}
%       \]
%       \item The \textbf{graph} of a function
%       \[
%         f: D \rightarrow \mathbb{R}, \quad y = f(x), \quad D \subseteq \mathbb{R}
%       \]
%       is given by the set
%       \[
%         \{(x,y) \in \mathbb{R}^2 \mid y = f(x), x \in D\}
%       \]
%       \item Curves in the plane can also be given by parametric equations:
%       \[
%         x = f(t), \quad y = g(t)
%       \]
%       where \( t \) is a parameter.
%     \end{itemize}

%   \subsection*{3-dimensional space}
%     \begin{itemize}
%       \item \textbf{3-dimensional space} can be represented by the set
%       \[
%         \mathbb{R}^3 = \{(x,y,z) \mid x,y,z \in \mathbb{R}\}
%       \]
%       \item \textbf{Right-handed system}: The \( x, y, z \) axes are a right-handed system. The positive \( x, y, z \) directions are determined by the right-hand rule:
%         \begin{enumerate}
%           \item Point the fingers of your right hand in the positive \( x \)-direction.
%           \item Curl your fingers in the positive \( y \)-direction.
%           \item Your thumb points in the positive \( z \)-direction.
%         \end{enumerate}
%     \end{itemize}

%   \subsection*{Curves in $\mathbb{R}^3$}
%     \begin{itemize}
%       \item Curves in \( \mathbb{R}^3 \) can be represented using parametric equations:
%       \[
%         x = f(t), \quad y = g(t), \quad z = h(t)
%       \]
      
%       \item There is \hl{no} way of turning these parametric equations of a curve in space intoa single Cartesion equation
%     \end{itemize}


%   \subsection*{Surfaces in $\mathbb{R}^3$}
%   \begin{itemize}
%     \item A \textbf{surface} in $\mathbb{R}^3$ is given by a single equation involving $x,y,x$
%     \item The general form of a plane is $ax + by + cz = d$
%     \item The general form of a \textbf{sphere} with radius $r$ and centre $(a,b,c)$ is $(x-a)^2 (y-b)^2 + (z-c)^2 = r^2$
%     \item The general form of a \textbf{paraboloid} is given by 
%     \begin{center}
%       $z = c \pm ((x-a)^2 + (y-b)^2)$
%     \end{center}
%   \end{itemize}

\section{Week8}
  \subsection*{Function of one variable}
  \begin{itemize}
  	\item \textbf{Definition}: Recall that a \hl{function of one real variable}
  		\begin{center}
  			$f: D \rightarrow \mathbb{R}, D \subseteq \mathbb{R}$
  		\end{center}
  		is a rule that assigns to each number $x \in D$ a number $f(x) \in \mathbb{R}$
  	\item The \textbf{domain} of $f$ is the \hl{set D of allowed inpus}.
  	\item The \textbf{natural domain} of $f$ is \hl{the largest subset of R of allowed inputs}.
  \end{itemize}

  \subsection*{Function of 2 variables}
  \begin{itemize}
  	\item \textbf{Definition:} A \hl{function of 2 real variables}:
  	\begin{center}
  		$f: D \rightarrow \mathbb{R}, D \subseteq \mathbb{R}^2$
  	\end{center}
  	is a \hl{rule that assigns to each pair $(x,y) \in D$ a number $f(x,y) \in \mathbb{R}$}

  	\item The \textbf{domain} of f is the \hl{set D of allowed inpus}

  	\item The \textbf{natrual domian of $f$} is the \hl{largest subset of $\mathbb{R}^2$ of allowed inputs}
  \end{itemize}

  \subsection*{Graphs of functions}
  \begin{itemize}
  	\item The \hl{graph} of a function of 2 variables:
  		\begin{center}
  			$f: D \rightarrow \mathbb{R}$
  		\end{center}
  		is the set of points
  		\begin{center}
  			$\{(x,y,f(x,y))\in \mathbb{R}^3 | (x,y) \in D\}$
  		\end{center}
		\includegraphics[width=1.0\textwidth, height=8cm]{Graphs/w8_1.png}

	\item We \hl{can not} get a full sphere as a function. It fails the vertical line test.
  \end{itemize}

  \subsection*{Level Curves}
  \begin{itemize}
  	\item \textbf{Definition:} A \hl{level cureve} of a function $f(x,y)$ is a curve in $\mathbb{R}^2$ defined by 
  	\begin{center}
  		$f(x,y) = c$
  	\end{center}
  	for a constant $c \in \mathbb{R}$

  	\item The level curves $f(x,y) = c$ are the intersections of the surface $z = f(x,y)$ with the planes $z=c$

	\includegraphics[width=\linewidth, height=6cm]{Graphs/w8_2.png}

	\includegraphics[width=\linewidth, height=6cm]{Graphs/w8_3.png}

	\includegraphics[width=\linewidth, height=6cm]{Graphs/w8_4.png}
  \end{itemize}

  \subsection*{Partial derivatives}
  \begin{itemize}
  	\item \textbf{Definition:} for a sufficiently smooth function of 2 variables
  	\begin{center}
  		$f: D \rightarrow \mathbb{R}, D \subseteq \mathbb{R}^2$
  	\end{center}
  	The \hl{partial derivative of $f$ with respect to $x$ at $(x,y) = (a,b)$ is:}
  	\begin{center}
  		$f_x(a, b) = \frac{\partial f}{\partial x}\Bigg|_{(x,y)=(a,b)} = \lim_{h \to 0} \frac{f(a+h,b) - f(a,b)}{h}$
  	\end{center}
  	and the \hl{partial derivative of $f$ with respect to $y$ at $(x,y) = (a,b)$} is 
  	\begin{center}
  		$f_x(a, b) = \frac{\partial f}{\partial y}\Bigg|_{(x,y)=(a,b)} = \lim_{h \to 0} \frac{f(a,b+h) - f(a,b)}{h}$
  	\end{center}

  	\item \textbf{Terminology}: If $f_x(a,b) = \frac{\partial f}{\partial x}\Bigg|_{(a,b)}$ exists \hl{for all $(a,b) \in D$}, then we say that $f$ \hl{is diffrentiable with respect to $x$ on D} and we write
  	\begin{center}
  		$f_x(x,y) = \frac{\partial f}{\partial x}(x,y)$
  	\end{center}
  	for the \hl{derivative function of f w.r.t. x.}

  	\item Similarly, If $f_y(a,b) = \frac{\partial f}{\partial y}\Bigg|_{(a,b)}$ exists \hl{for all $(a,b) \in D$}, then we say that $f$ \hl{is diffrentiable with respect to $y$ on D} and we write
  	\begin{center}
  		$f_y(x,y) = \frac{\partial f}{\partial y}(x,y)$
  	\end{center}
  	for the \hl{derivative function of f w.r.t. y.}

  	\item What do partial derivatives measure? For a sufficiently smooth function:
  	\begin{center}
  		$f: D \rightarrow \mathbb{R}, D \subset \mathbb{R}$
  	\end{center}
  	the partial derivatives $f_x = \frac{\partial f}{\partial x}$ measure the rate of change of $f$ on the $x$ direction. 

	\includegraphics[width=\linewidth, height=6cm]{Graphs/w8_5.png}


	\includegraphics[width=\linewidth, height=6cm]{Graphs/w8_6.png}

	\includegraphics[width=\linewidth, height=6cm]{Graphs/w8_7.png}

	\item Here we have \hl{the intersection of the surface $z=f(x,y)$ and the plane $y=b$} is the function of one variable given by 
	\begin{center}
		$g(x) = f(x,b)$
	\end{center}
	The \hl{gradient} of the tangent to this curve at $x=a$ is given by
	\begin{center}
		$g'(a) = \lim_{h \to 0} \frac{g(a+h) - g(a)}{h}$ \\
		$= \lim_{h \to 0} \frac{f(a+h, b) - f(a, b)}{h}$ \\
		$= f_x(a, b)$
	\end{center} 

	\item How do we calculate partial derivatives?

	\begin{itemize}
		\item To calculate $f_x = \frac{\partial f}{\partial x}$
		\begin{enumerate}
			\item Imagine $y$ is a constant
			\item Differentiate as a function of one variable $x$.
		\end{enumerate}

		\item To calculate $f_y = \frac{\partial f}{\partial y}$
		\begin{enumerate}
			\item Imagine $x$ is a constant
			\item Differentiate as a function of one variable $y$.
		\end{enumerate}
	\end{itemize}

  \end{itemize}

\end{document}

