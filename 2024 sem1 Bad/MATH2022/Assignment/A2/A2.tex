\documentclass[12pt]{article}

% UPDATE THESE PARAMETERS

\newcommand{\sid}{530157791}
\newcommand{\assignment}{2}


\NeedsTeXFormat{LaTeX2e}
\usepackage[osf]{mathpazo}
\usepackage[svgnames]{xcolor}
\usepackage[T1]{fontenc}
\usepackage{amsmath,amsthm,amsfonts,amssymb,mathtools}
\usepackage{hyperref,url}
\usepackage[margin=2.7cm,a4paper]{geometry}
\usepackage{tasks}
\usepackage{xstring}
\usepackage[tikz]{mdframed}
\usepackage{environ}
\usepackage{etoolbox}
\usepackage{fourier-orns}
\usepackage{kvoptions}
\usepackage[]{units}
\usepackage{url}%
\usepackage[normal]{subfigure}

% math config

\DeclarePairedDelimiter\ceil{\lceil}{\rceil}
\DeclarePairedDelimiter\abs{\lvert}{\rvert}
\DeclarePairedDelimiter\set{\{}{\}}


% headers

\usepackage{fancyhdr}
\addtolength{\headheight}{2.5pt}
\pagestyle{fancy}
\fancyhead{} 
\fancyhead[L]{\sc MATH2022} 
\renewcommand{\headrulewidth}{0.75pt}

% update these headers
\fancyhead[C]{\sc SID: \sid}
\fancyhead[R]{Assignment \assignment}


% pseudo-code typesetting configuration
\usepackage[tikz]{mdframed}
\usepackage[noend]{algpseudocode}

\newenvironment{pseudocode}
{
  \mdframed
  \algorithmic[1]
}
{
  \endalgorithmic
  \endmdframed
}

% solution 

\newcommand{\solution}[1]{\noindent \textbf{Solution #1.}}


\begin{document}

\solution{1}

\subsubsection*{(a)}

  We need to find $\lambda_1, \lambda_2, \lambda_3$ such that:
  \begin{center}
    $\lambda_1(1,0,3)+\lambda_2(2,1,8)+\lambda_3(1,-1,2)=(3,-5,4)$
  \end{center}
  written in augmented matrix:
  \[
  \begin{aligned}
  \left[\begin{array}{ccc|c}
  1 & 2 & 1 & 3 \\
  0 & 1 & -1 & -5 \\ 
  3 & 8 & 2 & 4
  \end{array}\right]
  \overset{\text{R}_3 = 3\text{R}_1 - \text{R}_3}{\sim}
  \left[\begin{array}{ccc|c}
  1 & 2 & 1 & 3 \\
  0 & 1 & -1 & -5 \\
  0 & -2 & 1 & 5
  \end{array}\right]
  \overset{\text{R}_3 = 2\text{R}_1 + \text{R}_3}{\sim}
  \left[\begin{array}{ccc|c}
  1 & 2 & 1 & 3 \\
  0 & 1 & -1 & -5 \\
  0 & 0 & -1 & -5
  \end{array}\right]
  \\\overset{\text{R}_2 = \text{R}_2 -\text{R}_3}{\sim}
  \left[\begin{array}{ccc|c}
  1 & 2 & 1 & 3 \\
  0 & 1 & 0 & 0 \\
  0 & 0 & -1 & -5
  \end{array}\right]
  \overset{\text{R}_1 = \text{R}_1 - 2\text{R}_2 + \text{R}_3, \text{R}_3 = -\text{R}_3}{\sim}
  \left[\begin{array}{ccc|c}
  1 & 0 & 0 & -2 \\
  0 & 1 & 0 & 0 \\
  0 & 0 & 1 & 5
  \end{array}\right]
  \end{aligned}
  \]


  So that, we have:
  $$
  \left\{\begin{array}{l}
  \lambda_1=-2 \\
  \lambda_2=0 \\
  \lambda_3=5
  \end{array}\right.
  $$
  Thus,$[V]_B = (-2,0,5)$

\subsubsection*{(b)}
  We need to find $\lambda_1, \lambda_2, \lambda_3$ such that:
  \begin{center}
    $\lambda_1 (1 + t^2)+\lambda_2(t + t^2)+\lambda_3(1+2t+t^2)=(1+4t+7t^2)$
  \end{center}
  which is
  \begin{center}
    $(\lambda_1 + \lambda_3) + (\lambda_2 + 2\lambda_3)t + (\lambda_1+\lambda_2+\lambda_3)t^2=(1+4t+7t^2)$
  \end{center}
  written in augmented matrix:
  \[
  \begin{array}{c}
  \left[\begin{array}{ccc|c}
  1 & 1 & 1 & 7 \\
  0 & 1 & 2 & 4 \\
  1 & 0 & 1 & 1
  \end{array}\right]
  \overset{\text{R}_3 = \text{R}_3 - \text{R}_1}{\sim}
  \left[\begin{array}{ccc|c}
  1 & 1 & 1 & 7 \\
  0 & 1 & 2 & 4 \\
  0 & -1 & 0 & -6
  \end{array}\right]
  \overset{\text{R}_3 = \text{R}_3 + \text{R}_2}{\sim} \\
  \left[\begin{array}{ccc|c}
  1 & 1 & 1 & 7 \\
  0 & 1 & 2 & 4 \\
  0 & 0 & 2 & -2
  \end{array}\right]
  \overset{\text{R}_2 = \text{R}_2 - \text{R}_3}{\sim}
  \left[\begin{array}{ccc|c}
  1 & 1 & 1 & 7 \\
  0 & 1 & 0 & 6 \\
  0 & 0 & 2 & -2
  \end{array}\right]
  \overset{\text{R}_3 = \frac{1}{2} \text{R}_3}{\sim}
  \left[\begin{array}{ccc|c}
  1 & 1 & 1 & 7 \\
  0 & 1 & 0 & 6 \\
  0 & 0 & 1 & -1
  \end{array}\right]
  \overset{\text{R}_1 = \text{R}_1 - \text{R}_2,\text{R}_1 = \text{R}_1 - \text{R}_3}{\sim} \\
  \left[\begin{array}{ccc|c}
  1 & 0 & 0 & 2 \\
  0 & 1 & 0 & 6 \\
  0 & 0 & 1 & -1
  \end{array}\right]
  \end{array}
  \]

  So that, we have:
  $$
  \left\{\begin{array}{l}
  \lambda_1=2 \\
  \lambda_2=6 \\
  \lambda_3=-1
  \end{array}\right.
  $$
  Thus,$[V]_B = (2,6,-1)$

\solution{2}

  \[
  \begin{array}{c}
  \left[\begin{array}{ccccc}
  1 & 1 & 6 & 2 & 6 \\
  4 & 1 & 4 & 2 & 5 \\
  5 & 2 & 3 & 5 & 0 \\
  3 & 4 & 6 & 2 & 4 \\
  1 & 2 & 1 & 4 & 3 
  \end{array}\right]
  \overset{\text{R}_2 = \text{R}_2 - \text{R}_4 - \text{R}_5,\text{R}_3 = \text{R}_3 - \text{R}_4 - 2\text{R}_5}{\sim}
  \left[\begin{array}{ccccc}
  1 & 1 & 6 & 2 & 6 \\
  0 & -5 & -3 & -4 & -2 \\
  0 & -6 & -5 & -5 & -3 \\
  3 & 4 & 6 & 2 & 4 \\
  1 & 2 & 1 & 4 & 3 
  \end{array}\right] \\
  \overset{\text{R}_4 = \text{R}_4 - 3\text{R}_5, \text{R}_5 = \text{R}_5 - \text{R}_1 }{\sim}
  \left[\begin{array}{ccccc}
  1 & 1 & 6 & 2 & 6 \\
  0 & -5 & -3 & -4 & -2 \\
  0 & -6 & -5 & -5 & -3 \\
  0 & -2 & 3 & -10 & 2 \\
  0 & 1 & -5 & 2 & -3 
  \end{array}\right]
  \overset{\text{R}_4 = \text{R}_2 + \text{R}_4, \text{R}_5 = \text{R}_3 - \text{R}_5 }{\sim}
  \left[\begin{array}{ccccc}
  1 & 1 & 6 & 2 & 6 \\
  0 & 2 & -3 & -4 & -2 \\
  0 & -6 & -5 & -5 & -3 \\
  0 & 0 & 0 & 0 & 0 \\
  0 & 0 & 0 & 0 & 0 
  \end{array}\right]
  \\ \overset{\text{R}_3 = 3\text{R}_2 + \text{R}_3}{\sim}
  \left[\begin{array}{ccccc}
  1 & 1 & 6 & 2 & 6 \\
  0 & 2 & -3 & -4 & -2 \\
  0 & 0 & 0 & -3 & -2 \\
  0 & 0 & 0 & 0 & 0 \\
  0 & 0 & 0 & 0 & 0 
  \end{array}\right]
  \overset{\text{R}_2 = 4\text{R}_2 , \text{R}_3 = 2\text{R}_3 }{\sim}
  \left[\begin{array}{ccccc}
  1 & 1 & 6 & 2 & 6 \\
  0 & 1 & 2 & 5 & 6 \\
  0 & 0 & 0 & 1 & 3 \\
  0 & 0 & 0 & 0 & 0 \\
  0 & 0 & 0 & 0 & 0 
  \end{array}\right]
  \overset{\text{R}_2 = \text{R}_2 - 5\text{R}_3}{\sim}
  \left[\begin{array}{ccccc}
  1 & 1 & 6 & 2 & 6 \\
  0 & 1 & 2 & 0 & 5 \\
  0 & 0 & 0 & 1 & 3 \\
  0 & 0 & 0 & 0 & 0 \\
  0 & 0 & 0 & 0 & 0 
  \end{array}\right]
  \\ \overset{\text{R}_1 = \text{R}_1 - \text{R}_2 - 2 \text{R}_3}{\sim}
  \left[\begin{array}{ccccc}
  1 & 0 & 4 & 0 & 2 \\
  0 & 1 & 2 & 0 & 5 \\
  0 & 0 & 0 & 1 & 3 \\
  0 & 0 & 0 & 0 & 0 \\
  0 & 0 & 0 & 0 & 0 
  \end{array}\right]
  \end{array}
  \]
  
  \subsubsection*{(a)}
  According to "Rank-Nullity Theorem", the rank of a matrix is equal to the number of its non-zero rows.So that the rank of this Matrix over $\textbf{Z}_7$ is 3, and the nullity is 2.

  \subsubsection*{(b)}

  To find the basis of the Null space for the given system of linear equations, we express all other variables as linear combinations of free variables.

  \[
  \begin{bmatrix}
  1 & 0 & 4 & 0 & 2 \\
  0 & 1 & 2 & 0 & 5 \\
  0 & 0 & 0 & 1 & 3 \\
  \end{bmatrix} 
  \begin{bmatrix}
  x_1 \\
  x_2 \\
  x_3 \\
  x_4 \\
  x_5 \\
  \end{bmatrix}
  =
  \begin{bmatrix}
  0 \\
  0 \\
  0 \\
  \end{bmatrix}
  \]

  From the matrix, we have:
  \begin{itemize}
      \item $x_1 = 3x_3 + 5x_5$
      \item $x_2 = 5x_3 + 2x_5$
      \item $x_4 = 4x_5$
  \end{itemize}

  Since $x_3$ and $x_5$ are free variables, we set them to 1 in turn, with all other free variables at 0, to find the basis vectors of the Null space.

  \begin{itemize}
      \item Setting $x_3 = 1$ and $x_5 = 0$, we substitute into the above equations to obtain one basis vector:
      \[
      \begin{bmatrix}
      x_1 \\
      x_2 \\
      x_3 \\
      x_4 \\
      x_5 \\
      \end{bmatrix}
      =
      \begin{bmatrix}
      3 \\
      5 \\
      1 \\
      0 \\
      0 \\
      \end{bmatrix}
      \]

      \item Setting $x_3 = 0$ and $x_5 = 1$, we substitute into the above equations to obtain another basis vector:
      \[
      \begin{bmatrix}
      x_1 \\
      x_2 \\
      x_3 \\
      x_4 \\
      x_5 \\
      \end{bmatrix}
      =
      \begin{bmatrix}
      5 \\
      2 \\
      0 \\
      4 \\
      1 \\
      \end{bmatrix}
      \]
  \end{itemize}

  Therefore, the basis of the Null space is:
  \[
  \left\{
  \begin{bmatrix}
  3 \\
  5 \\
  1 \\
  0 \\
  0 \\
  \end{bmatrix},
  \begin{bmatrix}
  5 \\
  2 \\
  0 \\
  4 \\
  1 \\
  \end{bmatrix}
  \right\}
  \]

\solution{3}
  Assume that there is a linearly dependent subset \( Y \subseteq X \) . Since \( Y = \{y_1, y_2, \dots, y_m\} \) is a subset of \( X = \{x_1, x_2, \dots, x_k\} \), there exist scalars \( c_1, c_2, \dots, c_m \), not all zero, such that:

  \begin{center}
  $
  c_1 y_1 + c_2 y_2 + \dots + c_m y_m = 0
  $
  \end{center}

  Each \( y_i \) in \( Y \) corresponds to some \( x_j \) in \( X \). We extend this linear combination to include all elements of \( X \) by assigning a coefficient of 0 to the vectors in \( X \) that are not in \( Y \). Hence, we have:

  \begin{center}
    $
    c_1 x_{i_1} + c_2 x_{i_2} + \dots + c_m x_{i_m} + 0 \cdot x_{j_1} + 0 \cdot x_{j_2} + \dots + 0 \cdot x_{j_{k-m}} = 0
    $
  \end{center}

  where \( x_{i1}, x_{i2}, \dots, x_{im} \) are the vectors in \( Y \), and \( x_{j1}, x_{j2}, \dots, x_{jk-m} \) are the vectors in \( X \) not included in \( Y \).

  Since \( X \) is linearly independent, the only solution to this equation is for all coefficients \( c_1, c_2, \dots, c_m \) and all zero coefficients to be exactly zero. This contradicts our assumption that the coefficients \( c_1, c_2, \dots, c_m \) are not all zero.

  Therefore, the assumption is false, and any non-empty subset \( Y \subseteq X \) must be linearly independent.


\solution{4}
  \subsubsection*{(a)}
    \begin{itemize}
      \item $S(p(x, y))= 3-4y+yx+2x^2$
      \item $S(p(x, y))= y^2 - 4y + 3yx -4x +x^2 -1$
    \end{itemize}

  \subsubsection*{(b)}

  To prove that \( V_S \) is a subspace, we need to  prove that it is non-empty, and closed under addition and scalar multiplication.

  \begin{itemize}
    \item non-emptiness

    The subspace \( V_S \) is non-empty because it contains at least the zero polynomial. The zero polynomial is symmetric under any variable swap since \( S(0) = 0 \). Therefore, \( 0 \in V_S \), confirming that \( V_S \) is non-empty.

    \item closure under Addition

    Let \( p(x, y) \) and \( q(x, y) \) be two polynomials in \( V_S \). By definition, \( S(p(x, y)) = p(x, y) \) and \( S(q(x, y)) = q(x, y) \). Consider the polynomial \( p(x, y) + q(x, y) \):
    \[
    S(p(x, y) + q(x, y)) = S(p(x, y)) + S(q(x, y)) = p(x, y) + q(x, y).
    \]
    Since \( S(p(x, y) + q(x, y)) = p(x, y) + q(x, y) \), we conclude that \( p(x, y) + q(x, y) \in V_S \). Hence, \( V_S \) is closed under addition.

    \item closure under Scalar Multiplication

    Let \( p(x, y) \) be a polynomial in \( V_S \) and let \( c \) be any real scalar. Since \( S(p(x, y)) = p(x, y) \), consider the scalar multiple \( c \cdot p(x, y) \):
    \[
    S(c \cdot p(x, y)) = c \cdot S(p(x, y)) = c \cdot p(x, y).
    \]
    Since \( S(c \cdot p(x, y)) = c \cdot p(x, y) \), it follows that \( c \cdot p(x, y) \in V_S \). Thus, \( V_S \) is closed under scalar multiplication.

  \end{itemize}
  Therefore, \( V_S \) is a subspace of \( V \) as it is non-empty, and closed under both addition and scalar multiplication.

  \subsubsection*{(c)}
    We first apply the variable swap \( S \) to \( p(x, y) \):
    \begin{center}
        \( S(p(x, y)) = a_0 + a_1y + a_2x + a_3y^2 + a_4yx + a_5x^2 \)
    \end{center}

    To ensure that \( p(x, y) \) is symmetric, we require:
    \begin{align*}
        a_1 &= a_2, \\
        a_3 &= a_5.
    \end{align*}

    The symmetric polynomials can then be expressed as:
    \[ p(x, y) = a_0 + a_1(x + y) + a_3(x^2 + y^2) + a_4xy \]

    The basis for \( V_S \) therefore consists of the following polynomials:
    \begin{center}
        \(\{1, x + y, x^2 + y^2, xy\}\)
    \end{center}

    Considering the equation \(\lambda_1 + \lambda_2 (x+y) + \lambda_3 (x^2+y^2) + \lambda_4 xy = 0\), where \(\lambda_1, \lambda_2, \lambda_3, \lambda_4 \in \mathbb{R}\), the only solution is \(\lambda_1 = \lambda_2 = \lambda_3 = \lambda_4 = 0\). This implies that the set is linearly independent.

    Thus, the basis for \( V_S \) is \(\{1, x+y, x^2+y^2, xy\}\), and the dimension of this space is 4.


\end{document}