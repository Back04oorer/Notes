\documentclass[12pt]{article}

% UPDATE THESE PARAMETERS

\newcommand{\sid}{530157791}
\newcommand{\assignment}{2}


\NeedsTeXFormat{LaTeX2e}
\usepackage[osf]{mathpazo}
\usepackage[svgnames]{xcolor}
\usepackage[T1]{fontenc}
\usepackage{amsmath,amsthm,amsfonts,amssymb,mathtools}
\usepackage{hyperref,url}
\usepackage[margin=2.7cm,a4paper]{geometry}
\usepackage{tasks}
\usepackage{xstring}
\usepackage[tikz]{mdframed}
\usepackage{environ}
\usepackage{etoolbox}
\usepackage{fourier-orns}
\usepackage{kvoptions}
\usepackage[]{units}
\usepackage{url}%
\usepackage[normal]{subfigure}

% math config

\DeclarePairedDelimiter\ceil{\lceil}{\rceil}
\DeclarePairedDelimiter\abs{\lvert}{\rvert}
\DeclarePairedDelimiter\set{\{}{\}}


% headers

\usepackage{fancyhdr}
\addtolength{\headheight}{2.5pt}
\pagestyle{fancy}
\fancyhead{} 
\fancyhead[L]{\sc MATH2022} 
\renewcommand{\headrulewidth}{0.75pt}

% update these headers
\fancyhead[C]{\sc SID: \sid}
\fancyhead[R]{Assignment \assignment}


% pseudo-code typesetting configuration
\usepackage[tikz]{mdframed}
\usepackage[noend]{algpseudocode}

\newenvironment{pseudocode}
{
  \mdframed
  \algorithmic[1]
}
{
  \endalgorithmic
  \endmdframed
}

% solution 

\newcommand{\solution}[1]{\noindent \textbf{Solution #1.}}


\begin{document}

\solution{1}

\subsubsection*{(a)}

  We need to find $\lambda_1, \lambda_2, \lambda_3$ such that:
  \begin{center}
    $\lambda_1(1,0,3)+\lambda_2(2,1,8)+\lambda_3(1,-1,2)=(3,-5,4)$
  \end{center}
  written in augument matrix:
  \[
  \begin{aligned}
  \left[\begin{array}{ccc|c}
  1 & 2 & 1 & 3 \\
  0 & 1 & -1 & -5 \\ 
  3 & 8 & 2 & 4
  \end{array}\right]
  \overset{\text{R}_3 = 3\text{R}_1 - \text{R}_3}{\sim}
  \left[\begin{array}{ccc|c}
  1 & 2 & 1 & 3 \\
  0 & 1 & -1 & -5 \\
  0 & -2 & 1 & 5
  \end{array}\right]
  \overset{\text{R}_3 = 2\text{R}_1 + \text{R}_3}{\sim}
  \left[\begin{array}{ccc|c}
  1 & 2 & 1 & 3 \\
  0 & 1 & -1 & -5 \\
  0 & 0 & -1 & -5
  \end{array}\right]
  \end{aligned}
  \]


  So that, we have:
  $$
  \left\{\begin{array}{l}
  \lambda_1=-2 \\
  \lambda_2=0 \\
  \lambda_3=5
  \end{array}\right.
  $$

\subsubsection*{(b)}
  We need to find $\lambda_1, \lambda_2, \lambda_3$ such that:
  \begin{center}
    $\lambda_1 (1 + t^2)+\lambda_2(t + t^2)+\lambda_3(1+2t+t^2)=(1+4t+7t^2)$
  \end{center}
  which is
  \begin{center}
    $(\lambda_1 + \lambda_3) + (\lambda_2 + 2\lambda_3)t + (\lambda_1+\lambda_2+\lambda_3)t^2=(1+4t+7t^2)$
  \end{center}
  written in augument matrix:
  \[
  \begin{array}{c}
  \left[\begin{array}{ccc|c}
  1 & 1 & 1 & 7 \\
  0 & 1 & 2 & 4 \\
  1 & 0 & 1 & 1
  \end{array}\right]
  \overset{\text{R}_3 = \text{R}_3 - \text{R}_1}{\sim}
  \left[\begin{array}{ccc|c}
  1 & 1 & 1 & 7 \\
  0 & 1 & 2 & 4 \\
  0 & -1 & 0 & -6
  \end{array}\right]
  \overset{\text{R}_3 = \text{R}_3 + \text{R}_2}{\sim} \\
  \left[\begin{array}{ccc|c}
  1 & 1 & 1 & 7 \\
  0 & 1 & 2 & 4 \\
  0 & 0 & 2 & -2
  \end{array}\right]
  \overset{\text{R}_2 = \text{R}_2 - \text{R}_3}{\sim}
  \left[\begin{array}{ccc|c}
  1 & 1 & 1 & 7 \\
  0 & 1 & 0 & 6 \\
  0 & 0 & 2 & -2
  \end{array}\right]
  \overset{\text{R}_3 = \frac{1}{2} \text{R}_3}{\sim}
  \left[\begin{array}{ccc|c}
  1 & 1 & 1 & 7 \\
  0 & 1 & 0 & 6 \\
  0 & 0 & 1 & -1
  \end{array}\right]
  \overset{\text{R}_1 = \text{R}_1 - \text{R}_2,\text{R}_1 = \text{R}_1 - \text{R}_3}{\sim} \\
  \left[\begin{array}{ccc|c}
  1 & 0 & 0 & 2 \\
  0 & 1 & 0 & 6 \\
  0 & 0 & 1 & -1
  \end{array}\right]
  \end{array}
  \]

  So that, we have:
  $$
  \left\{\begin{array}{l}
  \lambda_1=2 \\
  \lambda_2=6 \\
  \lambda_3=-1
  \end{array}\right.
  $$



\solution{2}

  \[
  \begin{array}{c}
  \left[\begin{array}{ccccc}
  1 & 1 & 6 & 2 & 6 \\
  4 & 1 & 4 & 2 & 5 \\
  5 & 2 & 3 & 5 & 0 \\
  3 & 4 & 6 & 2 & 4 \\
  1 & 2 & 1 & 4 & 3 
  \end{array}\right]
  \overset{\text{R}_2 = \text{R}_2 - \text{R}_4 - \text{R}_5,\text{R}_3 = \text{R}_3 - \text{R}_4 - 2\text{R}_5}{\sim}
  \left[\begin{array}{ccccc}
  1 & 1 & 6 & 2 & 6 \\
  0 & -5 & -3 & -4 & -2 \\
  0 & -6 & -5 & -5 & -7 \\
  3 & 4 & 6 & 2 & 4 \\
  1 & 2 & 1 & 4 & 3 
  \end{array}\right] \\
  \overset{\text{R}_4 = \text{R}_4 - 3\text{R}_5, \text{R}_5 = \text{R}_1 - \text{R}_5 }{\sim}
  \left[\begin{array}{ccccc}
  1 & 1 & 6 & 2 & 6 \\
  0 & -5 & -3 & -4 & -2 \\
  0 & -6 & -5 & -5 & -7 \\
  0 & -2 & 3 & -10 & -2 \\
  0 & -1 & 5 & -4 & 3 
  \end{array}\right]
  \overset{\text{R}_4 = \text{R}_2 + \text{R}_4, \text{R}_5 = \text{R}_3 + \text{R}_5 }{\sim}
  \left[\begin{array}{ccccc}
  1 & 1 & 6 & 2 & 6 \\
  0 & 2 & -3 & -4 & 2 \\
  0 & -6 & -5 & -5 & -7 \\
  0 & 0 & 0 & 0 & 0 \\
  0 & 0 & 0 & 5 & 3 
  \end{array}\right]
  \\ \overset{\text{R}_4 = \text{R}_2 + \text{R}_4, \text{R}_5 = \text{R}_3 + \text{R}_5 }{\sim}
  \left[\begin{array}{ccccc}
  1 & 1 & 6 & 2 & 6 \\
  0 & 2 & -3 & -4 & 2 \\
  0 & -6 & -5 & -5 & -7 \\
  0 & 0 & 0 & 0 & 0 \\
  0 & 0 & 0 & 5 & 3 
  \end{array}\right]
  \end{array}
  \]



\end{document}