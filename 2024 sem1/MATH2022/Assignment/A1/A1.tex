\documentclass[12pt]{article}
\usepackage[utf8]{inputenc}
\usepackage{amsmath}
\usepackage{amsfonts}
\usepackage{amssymb}
\usepackage{graphicx}
\usepackage[left=2cm,right=2cm,top=2cm,bottom=2cm]{geometry}

\title{MATH2022 Assignment 1}
\author{SID:530157791}
\date{\today}

\begin{document}

\maketitle

\section{Q1}
	In group $G$,for each element $\alpha \neq e$, there is a unique inverse $\alpha^{-1}$ such that $\alpha \alpha^{-1} = e$(\textit{Inverse Element}).Since G has an even number of elements, the total number of non-identity elements is odd.Which means that if we try to partition them into paris like $(\alpha,\alpha^{-1})$, there must be at least 1 element can not be paired with a distinct element.The unpaired element must therefore be its own inverse,implying $\alpha = \alpha^{-1}$.For this $\alpha$, it follows that $\alpha^2 = \alpha \alpha = e$

\section{Q2}
	\subsection*{(a)}
		$$
		\begin{aligned}
		\operatorname{det}(A) & =-2\left|\begin{array}{ll}
		1 & 1 \\
		1 & 2
		\end{array}\right|+4\left|\begin{array}{ll}
		1 & 3 \\
		1 & 1
		\end{array}\right| \\
		& =-2+4 \cdot(-2) \\
		& =-10 \\
		& =0 
		\end{aligned}
		$$
		Therefore, Matrix A is not invertible over $Z_5$

	\subsection*{(b)}
		$$
		\begin{aligned}
		& {\left[\begin{array}{lll|l}
		1 & 3 & 1 & 2 \\
		1 & 1 & 2 & 3 \\
		0 & 2 & 4 & 4
		\end{array}\right] } \\
		= & {\left[\begin{array}{lll|l}
		1 & 3 & 1 & 2 \\
		0 & 2 & 4 & 4 \\
		0 & 2 & 4 & 4
		\end{array}\right] } \\
		= & {\left[\begin{array}{lll|l}
		1 & 3 & 1 & 2 \\
		0 & 2 & 4 & 4 \\
		0 & 0 & 0 & 0
		\end{array}\right] }
		\end{aligned}
		$$
		Since $z$ can take any value in $Z_5 = {0,1,2,3,4}$,and for each $z$,there is exactly 1 solution for $x$ and $y$.Thus,there are 5 solutions for this system.

\section{Q3}
	\subsection*{(a)}
		True.To do the conjugation $\beta^{-1} \alpha \beta$,we just simply replace every $\alpha_i$  in each cycle with $\alpha_i \beta$, it doesn't change the number of elements in a permutation.
	\subsection*{(b)}
		False.If $C = \textbf{0}$,the statement "If $AC = BC$, then $A = B$" still holds true when $A \neq B$.
	\subsection*{(c)}
		\text{True. We have} \(\det(A) = \det(A^T)\)\text{. If } \(A^T = -A\)\text{, then } \(\det(A) = \det(-A)\)\text{, and since } \(\det(-A) = (-1)^n \det(A)\) \text{ for an } \(n \times n\) \text{ matrix, where } \(n = 3\) \text{ in this case, it follows that } \(\det(A) = -\det(A)\)\
		\\ which indicates that  \(\det(A) = 0\)\text{.}

\section{Q4}
	Assume that the eigenvalue of eigenvector $\textbf{v}$ is $\lambda$,so that by the description above, we have:
	\begin{center}
		$A \textbf{v} = \lambda \textbf{v}$
	\end{center}
	To prove that \(B\mathbf{v}\) is also an eigenvector of \(A\) with the same eigenvalue \(\lambda\) as \(\mathbf{v}\), we need to show that \(A(B\mathbf{v}) = \lambda (B\mathbf{v})\).

	\textbf{Proof}
	\begin{center}
		$A(B\textbf{v})$\\
		$=(AB)\textbf{v}$\\
		$=B(A\textbf{v})$\\
		$=B\lambda \textbf{v}$\\
		$= \lambda (B\textbf{v})$
	\end{center}


\end{document}