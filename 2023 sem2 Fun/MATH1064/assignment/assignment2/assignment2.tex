\documentclass[12pt,a4paper]{article}
\usepackage{amsmath, amssymb}
\usepackage{array}
\usepackage{tikz}

\title{MATH1064 Assignment2}
\author{SID:530157791}
\date{\today}

\begin{document}

\maketitle


\subsection*{Answer to question1}
(a) $(p,q) = (5,13)$ ,  $n = p \cdot q = 65$\\\\
(b)\\ 

(i)
\begin{enumerate}
\item All numbers less than n and divisible by p are not coprime to n since they share a factor with p.So we have:\\
$A = \{p,2p,3p, ......,(q-1)p\}$ and $|A| = q-1$

\item Similarly, all numbers less than n and divisible by q are not coprime to n.
$B =  \{q,2q,3q, ......,(p-1)q\}$ and and $|B| = p-1$

\item $|A \cap B| = 0$

\item Thus there are $(p - 1) + (q-1)$(Inclusion-Exclusion Principle) numbers are not coprime to n.

\item Therefore there are $pq - 1 - (p + q - 2) = (p - 1)(q - 1)$ numbers are coprime(gcd(k, n) = 1) to n when $0 < k < n$.
\end{enumerate}
\quad (ii) $ \varphi(n) = (5-1)(13-1) = 48$\\\\
(c) $e = 7$, public key : (7,65)






==================================================================================================
\newpage
==================================================================================================
\subsection*{Answer to question2}
In this case,I choose $d = 7$ such that  $7 \cdot 7 \equiv 1 (mod 48)$\\
Private key : (7, 65)


\subsection*{Answer to question3}
Imformation :
\begin{itemize}
	\item Public key : (7,65)
	\item Private key : (7,65)
	\item SID : 530157791
	\item Last 8 digit : 30157791($M = \{m_1,  m_2 , ..., m_8\}$ = \{3,0,1,5,7,7,9,1\})
	\item d = 7
	\item e = 7
\end{itemize}

(a) $c_i \equiv m_i^7$ (mod 65),$0 \leq c < n $
\begin{itemize}
\item $c_1 = 3^7 $mod 65 = 42
\item $c_2 = 0^7$mod 65 = 0
\item $c_3 = 1^7$mod 65 = 1
\item $c_4 = 5^7$mod 65 = 60
\item $c_5 = 7^7$mod 65 = 58
\item $c_6 = 7^7$mod 65 = 58
\item $c_7 = 9^7$mod 65 = 9
\item $c_8 = 1^7$mod 65 = 1
\end{itemize}


==================================================================================================
\newpage

(b)$m_i^{'} \equiv c_i^7$ (mod 65),$0 \leq m < n $ ,$1 \leq i \leq 8,i \in \mathbb{N}$
\begin{itemize}
\item $m_1^{'} = 42^7 $mod 65 = 3
\item $m_2^{'} = 0^7$mod 65 = 0
\item $m_3^{'} = 1^7$mod 65 = 1
\item $m_4^{'} = 60^7$mod 65 = 5
\item $m_5^{'} = 58^7$mod 65 = 7
\item $m_6^{'}= 58^7$mod 65 = 7
\item $m_7^{'} = 9^7$mod 65 = 9
\item $m_8^{'} = 1^7$mod 65 = 1
\end{itemize}

$M^{'} = \{3,0,1,5,7,7,9,1\}$\\

$M = M^{'}$


==================================================================================================
\newpage
==================================================================================================
\subsection*{Answer to question4}
(a) James's(my friend) public key : (7 ,  33)\\\\
(b) Perelman.
\begin{itemize}
\item P = 16
\item E = 5
\item R =18
\item E = 5
\item L = 12
\item M = 13
\item A = 1
\item N = 14
\end{itemize}
Then encode $M = \{16,5,18,5,12,13,1,14\}$ by public key (7,33).\\
Encoded messages $c_i \equiv m_i^7$ mod n,  $1 \leq i \leq 8,i \in \mathbb{N}$\\
$C = \{25,14,6,14,12,7,1,20\}$\\\\
(c)Encoded messages from James:\\
$CJ$ = $\{50, 16, 16, 60, 14, 57, 60, 9, 52, 60, 47\}$\\
My private key : (7,65)\\
So that $MJ_i \equiv CJ_i^7$ (mod 65):\\
$MJ = \{15, 16, 16, 5, 14, 8, 5, 9, 13, 5, 18\}$\\
Which is Oppenheimer!\\
==================================================================================================

\end{document}
